%%%
%
% File: CalculusI.tex
%
% This is Part 1 of the APEX Calculus Book.
%
%%%

%%%
%
% I do not understand this part. I will use the book document class throughout.
%
%%%
%\RequirePackage{ifthen}
%\newboolean{longpage}
%\setboolean{longpage}{false}
%\ifthenelse{\boolean{longpage}}%
%{\documentclass[10pt]{article}}%
%{\documentclass[10pt]{book}}

%%%
%
% This sets the documetclass to book.
%
%%%
\documentclass[10pt]{book} % I really want to use 12pt here. I will experiment with it later.

%%%
%
% Import the formatting information.
%
%%%
\usepackage{searsbook}

%%%
%
% The longpage boolean seems to be important. I suspect there may be a better option for setting paper sizes.
%
%%%
\newboolean{longpage}
\setboolean{longpage}{false}

\newboolean{printlabelname}
\setboolean{printlabelname}{false}
\ifthenelse{\boolean{printlabelname}}{\usepackage[notref,notcite]{showkeys}}{}
\usepackage{pdfpages}


%% end detour
%\usepackage{amsmath}
\usepackage{xr}
\externaldocument{Calculus}


%%%
%
% It looks like the book was specifically tied to the page size. I suspect the authors were up to no good with alignments.
% I may be a while untangling this part.
%
%%%
%\input{headers/Page_Size_Calculus}

\usepackage{APEX_format}
\input{headers/Header_Calculus}
\usepackage{pgfplots}
\pgfplotsset{compat=1.8}
\usepackage{pdfpages}


%%%
%
% This is the original font information.
%
%%%
%\ifthenelse{\boolean{xetex}}%
%	{\sffamily
%	%%\usepackage{fontspec}
%	\usepackage{mathspec}
%	\setallmainfonts[Mapping=tex-text]{Calibri}
%	\setmainfont[Mapping=tex-text]{Calibri}
%	\setsansfont[Mapping=tex-text]{Calibri}
%	\setmathsfont(Greek){[cmmi10]}}
%	{}
%	
%	\ifthenelse{\boolean{luatex}}%
%	{\sffamily
%	\usepackage{fontspec}
%	\usepackage{unicode-math}
%	%\usepackage{mathspec}
%	%\setallmainfonts[Mapping=tex-text]{Calibri}
%	\setmainfont{Calibri}
%	%\setsansfont[Mapping=tex-text]{Calibri}
%	\setmathfont[range=\mathup]{Calibri}
%	\setmathfont[range=\mathit]{Calibri Italic}
%	}
%	{}

\makeindex

%%%\tracingonline=1
\begin{document}
%\printexercisenames
%\printincolor
\printinblackandwhite
%\printallanswers


\input{text/front_matter_and_coverI}


\chapter{Limits}\label{chapter:limits}
\thispagestyle{empty}
%
\input{text/01_Limit_Introduction}
\input{text/01_Limit_Definition}
\input{text/01_Analytic_Limits}
\input{text/01_One_Sided_Limits}
\input{text/01_Continuity}
\input{text/01_Limits_Involving_Infinity}

%
%%%\addtocounter{chapter}{1}
%
\clearpage{\pagestyle{empty}\cleardoublepage}
\chapter{Derivatives}\label{chapter:derivatives}
\thispagestyle{empty}
\addtocontents{toc}{\protect\enlargethispage{4\baselineskip}}

\input{text/02_Derivative}
\input{text/02_Derivative_Meaning}
\input{text/02_Derivative_Rules}
\input{text/02_Product_Quotient_Rules}
\input{text/02_Chain_Rule}
\input{text/02_Implicit_Differentiation}
\input{text/02_Derivative_Inverse_Functions}

%%\addtocounter{chapter}{2}

\clearpage{\pagestyle{empty}\cleardoublepage}
\chapter{The Graphical Behavior of Functions}\label{chapter:graphbehavior}
\thispagestyle{empty}

\input{text/03_Extreme_Values}
\input{text/03_Mean_Value_Theorem}
\input{text/03_Increasing_Decreasing}
\input{text/03_Concavity}
\input{text/03_Curve_Sketching}

%%\addtocounter{chapter}{3}

\clearpage{\pagestyle{empty}\cleardoublepage}
\chapter{Applications of the Derivative}\label{chapter:deriv_apps}
\thispagestyle{empty}

\input{text/04_NewtonsMethod}
\input{text/04_Related_Rates}
\input{text/04_Optimization}
\input{text/04_Differentials}

%
%%%\addtocounter{chapter}{4}
%\addtocontents{toc}{\protect\textbf{(Calculus II begins here)}\protect\par}

\clearpage{\pagestyle{empty}\cleardoublepage}
\chapter{Integration}\label{chapter:integration}
\thispagestyle{empty}
\addtocontents{toc}{\protect\thispagestyle{empty}}
\addtocontents{toc}{\protect\enlargethispage{4\baselineskip}}

\input{text/05_Antiderivatives}
\addtocontents{toc}{\protect\thispagestyle{empty}}
\input{text/05_Definite_Integral}
\input{text/05_Riemann_Sums}
\input{text/05_FTC}
\input{text/05_Numerical_Integration}
%
%%
%%%\addtocounter{chapter}{5}
%%
\clearpage{\pagestyle{empty}\cleardoublepage}
\chapter{Techniques of Antidifferentiation}\label{chapter:anti_tech}
\thispagestyle{empty}
\addtocontents{toc}{\protect\thispagestyle{empty}}

\input{text/06_Substitution}
%\input{text/06_Int_By_Parts}
%\input{text/06_Trigonometric_Integrals}
%\input{text/06_Trig_Sub}
%\input{text/06_Partial_Fractions}
%\input{text/06_Hyperbolic_Functions}
%\input{text/06_LHopitals_Rule}
%\input{text/06_Improper_Integration}
%
%%%%\addtocounter{chapter}{6}
%
%\clearpage{\pagestyle{empty}\cleardoublepage}
%\chapter{Applications of Integration}\label{chapter:app_of_int}
%\thispagestyle{empty}
%
%\input{text/07_Area_Between_Curves}
%\input{text/07_Disk_Washer_Method}
%\input{text/07_Shell_Method}
%\input{text/07_Arc_Length}
%\input{text/07_Work}
%\input{text/07_Fluid_Force}
%
%%%%\addtocounter{chapter}{7}
%\clearpage{\pagestyle{empty}\cleardoublepage}
%\chapter{Sequences and Series}\label{chapter:sequences_series}
%\thispagestyle{empty}
%\addtocontents{toc}{\protect\thispagestyle{empty}}
%
%\input{text/08_Sequences}
%\input{text/08_Series}
%\input{text/08_Integral_Comparison_Tests}
%\input{text/08_Ratio_Root_Tests}
%\input{text/08_Alternating_Series}
%\input{text/08_Power_Series}
%\input{text/08_Taylor_Polynomials}
%\input{text/08_Taylor_Series}
%%%
%%%%\addtocounter{chapter}{8}
%\clearpage{\pagestyle{empty}\cleardoublepage}
%\chapter{Curves in the Plane}\label{chapter:planar_curves}
%\thispagestyle{empty}
%
%\input{text/09_Conic_Sections}
%\input{text/09_Parametric_Equations}
%\input{text/09_Parametric_Calculus}
%\input{text/09_Polar_Intro}
%\input{text/09_Polar_Calculus}
%
%
%\addtocontents{toc}{\protect\clearpage}
%%
%%
%%%\addtocounter{chapter}{9}
%\clearpage{\pagestyle{empty}\cleardoublepage}
%\chapter{Vectors}\label{chapter:vectors}
%\thispagestyle{empty}
%\input{text/10_Space_Intro}
%\input{text/10_Vector_Introduction}
%\input{text/10_Dot_Product}
%\input{text/10_Cross_Product}
%\input{text/10_Lines}
%\input{text/10_Planes}
%
%
%\addtocontents{toc}{\protect\thispagestyle{empty}}
%\addtocontents{toc}{\protect\enlargethispage{7\baselineskip}}
%
%%%%\addtocounter{chapter}{10}
%\clearpage{\pagestyle{empty}\cleardoublepage}
%\chapter{Vector Valued Functions}\label{chap:vvf}
%\thispagestyle{empty}
%\input{text/11_Vector_Functions_Intro}
%\input{text/11_Vector_Functions_Calc}
%\input{text/11_Vector_Functions_Motion}
%\input{text/11_Vector_Tangent_Normal}
%\input{text/11_Arc_Length_Parameter_Curvature}
%
%\clearpage{\pagestyle{empty}\cleardoublepage}
%\chapter{Functions of Several Variables}\label{chapter:multi}
%\thispagestyle{empty}
%\input{text/12_Multivariable_Intro}
%\input{text/12_Multivariable_Limit}
%\input{text/12_Partial_Derivatives}
%\input{text/12_Total_Differential}
%\input{text/12_Multivariable_Chain_Rule}
%\input{text/12_Directional_Derivatives}
%\input{text/12_Tangent_Planes_Lines}
%\input{text/12_Extrema}
%
%%%\addtocounter{chapter}{12}
%\clearpage{\pagestyle{empty}\cleardoublepage}
%\chapter{Multiple Integration}\label{chapter:mult_int}
%\thispagestyle{empty}
%\input{text/13_Iterated_Integrals}
%\input{text/13_Double_Integrals_Volume}
%\input{text/13_Double_Integrals_Polar}
%\input{text/13_Center_of_Mass}
%\input{text/13_Surface_Area}
%\input{text/13_Triple_Integration}
%\input{text/13_Cylindrical_Spherical}
%
%
%\clearpage{\pagestyle{empty}\cleardoublepage}
%\chapter{Vector Analysis}\label{chapter:vector_calc}
%\thispagestyle{empty}
%\input{text/14_Line_Integral_Intro}
%\input{text/14_Vector_Fields}
%\input{text/14_Line_Integral_Vector_Fields}
%\input{text/14_Line_Integral_Greens_Theorem}
%\input{text/14_Parametrized_Surfaces}
%\input{text/14_Surface_Integrals}
%\input{text/14_Stokes_Divergence}

%%%\layout


%\clearpage



\appendix
\clearpage{\pagestyle{empty}\cleardoublepage}

\pagenumbering{arabic}\renewcommand{\thepage}{A.\arabic{page}}
\makeexercisesection{Solutions To Selected Problems}




\qendheader
\eendgeometry

\phantomsection
\cleardoublepage
\addcontentsline{toc}{chapter}{\indexname}
\printindex

\cleardoublepage
\input{text/Inside_Cover_Of_The_Text_Material_Complete}


\end{document}

